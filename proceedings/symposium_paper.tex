% iaus2esa.tex -- sample pages for Proceedings IAU Symposium document class
% (based on v1.0 cca2esam.tex)
% v1.04 released 17 May 2004 by TechBooks
%% small changes and additions made by KAvdH/IAU 4 June 2004
% Copyright (2004) International Astronomical Union

\NeedsTeXFormat{LaTeX2e}

\documentclass{iau} 
\usepackage{graphicx}
\usepackage{hyperref}


\title[Lessons learned by the Andean ROAD] %% give here short title %%
{Lessons learned by the Andean ROAD}

\author[J. E. Forero-Romero]   %% give here short author list %%
{Jaime E. Forero-Romero$^1$,
\'Angela Patricia P\'erez Henao$^2$ 
%%  \thanks{Present address: Fluid Mech Inc., 24 The Street, Lagos, Nigeria.},
 \and Germ\'an Chaparro$^3$}

\affiliation{$^1$Departamento de F\'isica, Universidad de los Andes,
  \\ Calle 18A No. 1 - 10, Bogot\'a, Colombia \\ email: {\tt
    je.forero@uniandes.edu.co} \\[\affilskip] 
$^2$Planetario de Medell\'in, \\ 
Carrera 52 No. 71 - 117, Medell\'in, Colombia\\
email: {\tt angela.perez@parqueexplora.org} \\[\affilskip]
$^2$Dept. of Astronomy \& Space Physics, Uppsala University, \\ Box
515, SE-75120 Uppsala, Sweden \\email: {\tt hoefner@astro.uu.se}}

\pubyear{2015}
\volume{xxx}  %% insert here IAU Symposium No.
\setcounter{page}{1}
\jname{Title of your IAU Symposium}
\editors{A.C. Editor, B.D. Editor \& C.E. Editor, eds.}
\begin{document}

\maketitle

\begin{abstract}
The Andean Regional Office of Astronomy for Development is now three
years old. The structure of the Office replicates the target groups in
the Central Office into three groups Universities (Task Force 1),
Schools (Task Force 2) and the General Public (Task Force 3).  In 2013
we gathered with tthe potential stakeholders in the region to define a
work plan 2015-2020 on these two fronts. We are now ready to share
some of the lessons learned.  
Here we summarize our sucesses and biggest challenges, hoping to
provide a new point of view on how Astronomy is helping, and sometimes
redefining, the Development Agenda. 

\keywords{Developement, Post-Development}
%% add here a maximum of 10 keywords, to be taken form the file <Keywords.txt>
\end{abstract}

\firstsection % if your document starts with a section,
              % remove some space above using this command.
\section{Introduction}

%\cite[Anders \& Zinner (1993)]{AndersZinner93} and 
%\cite[Ott (1993)]{Ott93}.

The Andean Regional Office of Astronomy for Development started its
activities in 2013 and was officially signed into existence in 2015. 
After three years of operations we have deployed different activities
first outlined in our original
proposal \footnote{\url{https://github.com/AndeanROAD/ProposalROAD}}. 
Here we present our structure, successes, challenges and outlook for
the next two years. 

\section{The Andean ROAD structure}


Its coordinations is a shared responsibility between Los Andes
University (Colombia), Parque Explora (Colombia) and the Chilean
Astronomical Society.
The structure of the office defined its strategy with three different
targets: Universities, Schools and the General Public. 



\section{Successes}

The most important success has been keeping a conversation with the
central OAD office and the IAU memeber interested on development
activities. 
The networks created in such conversations have helped us to keep
motivated and define new strategies.
\begin{itemize}
\item Organizing two regional schools aimed at advanced undegraduate
  and graduate students. The first one was held in Quito (2014) with a
  on astroparticle physics. The second one was held in Bogot\'a (2015)
  with a focus on cosmology.
\item Organizing two regional meetings aimed at defining the structure
  and projects to be developed by the Andean ROAD. The two meetings
  have been held in Bogot\'a (2013, 2015). 
\item Developing didactic material for visually impaired people
(project \emph{Astronom\'ia con todos los sentidos}). This
  material is currently being used across Colombia and Chile.
\end{itemize}



\section{Challenges}

The main challenge has been keeping the activities through volunteers
and without permanent funding. 
After this, the challenge across all our lines of work has been
finding better ways to share across institutions the lessons learned
and produced materials. 

We also lost the coordinator for the General Public activities, this
has left us with only two active lines of work: Universities and
Schools. 

Another meta-challenge has been trying to define what development means for
us in our region. 
The mainstream development concept used by the OAD is still very much
influenced by ideologies from the global north.
Postdevelopment concepts (based on a decolonial and more balanced
conversation) although more interesting and relevant to our realities
(\cite{Grosfoguel02}), are harder to bring into the conversation of
hierarchical organizations (IAU, OAD, ROADs).  

\section{Outlook}
The the next two years the Andean ROAD will continue its work with two
important goals in mind: organizing two regional meetings (Per\'u
and Bolvia) to discuss strategies to stablish astronomy (at the research,
teaching and outreach level) as a tool for development and
consolidating the production of materials for the \emph{Astronom\'ia
  con todos los sentidos} program.  

\begin{thebibliography}{}

\bibitem[Grosfoguel (2002)]{Grosfoguel02}
{Grosfoguel, R. (2002)}, {Colonial Difference, Geopolitics of Knowledge
  and Global Coloniality 
  in the Modern/Colonial Capitalist World System}, 
\textit{Review} 19, 2, pp. 131-154.


\end{thebibliography}

\end{document}
